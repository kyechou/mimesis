%-------------------------------------------------------------------------------
\section{Background}
%-------------------------------------------------------------------------------

In this section, we introduce some middleboxes and binary analysis tools that
we have surveyed for this project, including three software middleboxes, \ca
HAProxy, \cb Nginx, \cc Snort, and two binary analysis tools, \ca BAP and \cb
angr.


\subsection{Middleboxes}
%-------------------------------------------------------------------------------

Middleboxes (or network functions) are a set of network devices that are widely
used in various network designs. These devices provide different functions for
controlling packets rather than simple packet forwarding, usually in the middle
of network datapaths between the source hosts and destination hosts. According
to RFC 3234~\cite{rfc3234}, common middleboxes can be categorized into several
classes, including firewall filtering, network address translation, load
balancer, \etc. In this semester, we took a look at some popular,
production-grade software middleboxes, and tried to understand how they work and
how we should extract the mappings using analysis tools.

\subsubsection{HAProxy}

HAProxy is a connection-terminating load balancer, which means that an HAProxy
instance will terminate an incoming TCP connection and re-initiate the
connection on the other side(s), or drop the connection, according to the
configuration. There are essentially two modes in HAProxy, TCP and HTTP. When
configured in TCP mode, HAProxy would still terminate the TCP connection locally
and start new connections to the backend servers, which hence normalizes the TCP
packets (i.e., discarding or replacing those out-of-order packets or packets
with abnormal flags). The difference between the TCP and HTTP mode is that in
HTTP mode, HAProxy would inspect the layer 7 header and can be configured using
those HTTP information in the HTTP request header (e.g., HTTP request URI),
while in TCP mode, HAProxy will only look at the layer 3 and 4 information,
which can be useful for better performance or encrypted payload like HTTPS
without keys for decryption.

The non-deterministic behaviors of HAProxy given a concrete configuration can be
defined as the set of possible outcomes or actions that cannot be
deterministically modelled by only knowing the input request. From the
documentation of the current stable release (2.0.8), the supported load
balancing schemes include round-robin, least connections first, first available
first, source address hashing, HTTP URI hashing, HTTP URI parameters, and
random.

%\subsubsection{Nginx}
%
%Nginx plays two roles in network: reverse proxy and load balancer. These two
%roles are not independent. For reverse proxy, two modules are used: \ca Proxy
%module and \cb rewrite module. Possible non-deterministic actions from these two
%modules are:
%\begin{itemize}
%    \item proxy\_set\_header: redefine the value of certain field in the header
%        part of http request.
%    \item proxy\_set\_body: redefine the request body passed to the proxied
%        server.
%    \item proxy\_redirect: set the text that should be changed in the "Location"
%        and "Refresh" header fields of a proxied server response.
%    \item proxy\_pass: Set the protocol and address of a proxied server and an
%        optional URI to which a location should be mapped.
%	\item rewrite: redirect; changes the URI in the header.
%\end{itemize}
%Among the modules, three actions, `proxy\_set\_header', `rewrite' and
%`proxy\_pass', are the most possible functions to affect the output packet
%equivalence classes.
%
%For load balancer, upstream module is used. After analyze the module, we find
%one ground truth for load balancer is that it will definitely modify the packet
%header. However, the forwarding behavior is not affected. As Nginx only performs
%HTTP load balancing based on reverse proxy, HTTP headers might be modified
%during this process based on the algorithms that user chose. For example, hash
%methods will make sure packets from a certain IP address can only be forwarded
%to a certain backend server. While round-robin or fair methods will distribute
%all the packets to different backend servers. Thus, checking the modified HTTP
%packets can be a reasonable direction for analyzing non-deterministic behaviors.

\subsubsection{Snort}

Snort is an intrusion detection and prevention system (IDS/IPS), which is
usually placed near a firewall as an additional protection for untrusted
traffic. There are several modes provided by Snort which make it easier for the
users to switch the settings. For example, Sniffer mode and packet logger mode
are the two modes that don't change the packet equivalence classes, while NIDS
mode (network intrusion detection system mode), on the other hand, is much more
interesting since it performs detection and analysis on network traffic. Apart
from acting as a stateless firewall, Snort's "\textbf{flow}" keyword, used in
conjunction with session tracking, allows Snort to track and take action on TCP
packets depending on the TCP session state and its flag value. This
functionality gives network operators fine-grained control over Snort's
behavior.


\subsection{Binary analysis tools}
%-------------------------------------------------------------------------------

Binary analysis tools are used for analyzing programs at the level of binary
code, mostly for security threat assessment and vulnerability testing. Since
many software network functions are close-sourced and owned by companies, in
this project we try to analyze the middleboxes at the binary level, assuming
that we have the binary executables of the program. We looked at two binary
analysis tools, BAP and angr, during this semester, and then decided to try out
angr for this course project.

\subsubsection{BAP}

BAP is a program analysis platform written in OCaml by a team from CMU, but it
also comes with Python and C binding libraries, although there are not much
documentation about the APIs and binding libraries. BAP (or other similar binary
analysis tools) basically disassembles the binary code and lifts it into their
own RISC-like BAP Instruction Language (BIL) representation, on which program
analysis is performed. We have explored a few of BAP's core functions, such as
constructing control flow graph (CFG) and call graph (CG) given a program
binary. Control flow graph is a directed graph where each node is a code block
of sequential instructions without branches, except for the beginning and the
end of the block, and each edge is a branch from one block to another. Call
graph, nonetheless, is a directed graph consisting of code blocks of functions
with edges being function calls or returning to the caller function.

\subsubsection{angr}

angr is another popular binary analysis tool written in Python by a team from
UCSB and Arizona State University. angr shares many similar functions with BAP,
including disassembling program binaries, constructing control flow graph and
call graph, etc. angr also supports concolic execution analysis, which is useful
for our purpose, and some other functions that may not be helpful for our work,
such as automatic ROP (return-oriented programming) chain building. angr is able
to symbolically execute the target program, track the execution path with
execution "states", which are essentially sets of the execution contexts,
including the registers and main memory, and evaluate the symbolic expressions
with variables and constraints using Z3~\cite{z3} as the solver backend.
