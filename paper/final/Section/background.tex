\section{Background}

\subsection{Assumptions on middleboxes}

% describe the middleboxes implementation (connection terminating and
% forwarding)
After the survey of various open source middleboxes, we observed that, in
general, the middleboxes can be categorized into two classes. One of them is the
middleboxes that would forward the transport layer header (assuming that the
end-to-end application is based on some connection-oriented protocol, like TCP),
such as NetFilter subsystem in Linux and its derivative utilities (e.g.,
iptables, nftables). These middleboxes might still forward packets to different
next hops or modify the packet headers, but they will not modify the packets so
that it breaks the transport layer connection between the client and the server.
A common example is using iptables to configure the NetFilter subsystem as a
firewall and/or NAT, in which case the layer 2 and layer 3 headers are modified,
but layer 4 header fields remain untouched. The other category consists of the
middleboxes that terminate the incoming connection at one end and start a new
connection to the backend server at another. Such middleboxes usually act as a
proxy or gateway, such as TCP proxies, HTTP proxies, SOCKS proxies (layer 5),
etc. To simplify the problem, we will only focus on the second kind of
middleboxes, which would accept/terminate the incoming connection and initiate a
new one, so that we treat each connection as request and reply application-layer
packets.

\subsection{Challenges and design of example middlebox}

% explain why it's hard to analyze such implementations
%   function pointers, event-driven, modular design, multi-threading
To analyze a given middlebox software binary, however, can be quite challenging
for several reasons. One major reason is that many of these production-grade
middlebox software are highly optimized in order to achieve high performance
within software and hence drastically increase the complexity of binary
analysis, and oftentimes these optimizations do not affect the eventual routing
behavior. For example, we looked into the implementation of HAProxy, which is
both a TCP proxy and an HTTP proxy depending on the configuration. The
implementation is highly modularized, object-oriented (function pointers),
event-driven, and multi-threaded, so that it is difficult to figure out where
the routing decision logic happens in the program, track how the variables and
data structures are passed, and localize the caller of a certain function. While
looking at the source code, we identified the part where they accept the
incoming connections, but we couldn't track the returned data structures without
decent documentations or the help of binary analysis tools.

% why we build our own implementation of example middlebox
% design overview of the example middlebox
In addition, since we are trying to analyze the binaries for the sake of
verifying the functional behavior of the middleboxes rather than their
performance, we implemented our own version of a load balancer prototype with
essential functionalities but without performance optimizations, so as to
simplify the analysis. The simple load balancer prototype currently supports
basic TCP load balancing schemes like round-robin and least-connection-first,
and we plan to add more interesting load balancing algorithms in the future,
possibly source-based hashing, HTTP URL hashing, etc. The middlebox prototype
follows the typical POSIX network application implementation. It first opens up
a socket, binds, and listens at the port. Once a connection is accepted, it then
forks a child process for the connection and keeps the states of both the
connections and the pre-configured backend servers. The forked child processes
will then act as proxies and initiate connections to the actual servers based on
the load balancing schemes mentioned above. The prototype will be used as a
straw man example and a proof of concept. If it goes well and time permits, we
will then move on to more complicated middlebox implementations.

\subsection{Binary analysis tools and analysis procedure}

% explain that we don't have time to learn bap and ocaml in this course
% so we chose angr for now.
We mentioned in the previous report that we were looking at binary analysis
tools like BAP and angr. However, after spending a few weeks on BAP, we think
that, for this course project, we don't really have that much time to learn a
new programming language (OCaml) for BAP, and without a good understanding of
the language, it is impossible to understand their documentations or APIs. As a
result, we will only focus on angr, which is written in Python, for the course
project, although BAP seems to have better performance. We may explore more
about BAP and learn OCaml properly after this course.
