%%%%%%%%%%%%%%%%%%%%%%%%%%%%%%%%%%%%%%%%%%%%%%%%%%%%%%%%%%%%%%%%%%%%%%%%%%%%%%%%
% Template for USENIX papers.
%
% History:
%
% - TEMPLATE for Usenix papers, specifically to meet requirements of
%   USENIX '05. originally a template for producing IEEE-format
%   articles using LaTeX. written by Matthew Ward, CS Department,
%   Worcester Polytechnic Institute. adapted by David Beazley for his
%   excellent SWIG paper in Proceedings, Tcl 96. turned into a
%   smartass generic template by De Clarke, with thanks to both the
%   above pioneers. Use at your own risk. Complaints to /dev/null.
%   Make it two column with no page numbering, default is 10 point.
%
% - Munged by Fred Douglis <douglis@research.att.com> 10/97 to
%   separate the .sty file from the LaTeX source template, so that
%   people can more easily include the .sty file into an existing
%   document. Also changed to more closely follow the style guidelines
%   as represented by the Word sample file.
%
% - Note that since 2010, USENIX does not require endnotes. If you
%   want foot of page notes, don't include the endnotes package in the
%   usepackage command, below.
% - This version uses the latex2e styles, not the very ancient 2.09
%   stuff.
%
% - Updated July 2018: Text block size changed from 6.5" to 7"
%
% - Updated Dec 2018 for ATC'19:
%
%   * Revised text to pass HotCRP's auto-formatting check, with
%     hotcrp.settings.submission_form.body_font_size=10pt, and
%     hotcrp.settings.submission_form.line_height=12pt
%
%   * Switched from \endnote-s to \footnote-s to match Usenix's policy.
%
%   * \section* => \begin{abstract} ... \end{abstract}
%
%   * Make template self-contained in terms of bibtex entires, to allow
%     this file to be compiled. (And changing refs style to 'plain'.)
%
%   * Make template self-contained in terms of figures, to
%     allow this file to be compiled.
%
%   * Added packages for hyperref, embedding fonts, and improving
%     appearance.
%
%   * Removed outdated text.
%
%%%%%%%%%%%%%%%%%%%%%%%%%%%%%%%%%%%%%%%%%%%%%%%%%%%%%%%%%%%%%%%%%%%%%%%%%%%%%%%%

\documentclass[letterpaper,twocolumn,10pt]{article}
\usepackage{usenix2019_v3}

% to be able to draw some self-contained figs
\usepackage{tikz}
\usepackage{amsmath}

% inlined bib file
\usepackage{filecontents}

\usepackage{booktabs} % Top and bottom rules for tables

%-------------------------------------------------------------------------------
%   CODE INCLUSION CONFIGURATION
%-------------------------------------------------------------------------------

\usepackage{listings}

\lstset{language=C,
        frame=L,
        belowcaptionskip=1\baselineskip,
        breaklines=true,
        xleftmargin=\parindent,
        showstringspaces=false,
        basicstyle=\small\ttfamily,
        keywordstyle=\bfseries\color{green!60!black},
        commentstyle=\color{gray},
        identifierstyle=\color{black},
        stringstyle=\color{orange!70!black},
        %numbers=left, % Line numbers on left
        %firstnumber=1, % Line numbers start with line 1
        %numberstyle=\scriptsize\ttfamily\color{Brown},
}

\newcommand{\eg}{{\it e.g.,}\xspace}
\newcommand{\viz}{{\it viz.,}}
\newcommand{\Eg}{{\it E.g., }}
\newcommand{\etal}{{\it et~al.}}
\newcommand{\ie}{{\it i.e.,}\xspace}
\newcommand{\etc}{{\it etc.~}}
\newcommand{\ci}{{\it (i) }}
\newcommand{\cii}{{\it (ii) }}
\newcommand{\ciii}{{\it (iii) }}
\newcommand{\civ}{{\it (iv) }}
\newcommand{\cv}{{\it (v) }}
\newcommand{\cvi}{{\it (vi) }}
\newcommand{\ca}{{\it (a) }}
\newcommand{\cb}{{\it (b) }}
\newcommand{\cc}{{\it (c) }}
\newcommand{\cd}{{\it (d) }}
\newcommand{\ce}{{\it (e) }}
\newcommand{\cf}{{\it (f) }}
\newcommand{\cg}{{\it (g) }}
\newcommand{\ch}{{\it (h) }}
%\newcommand{\ci}{{\it (i) }}
\newcommand{\cj}{{\it (j) }}
\newcommand{\ck}{{\it (k) }}
\newcommand{\cl}{{\it (l) }}
\newcommand{\cm}{{\it (m) }}
\newcommand{\cn}{{\it (n) }}
\newcommand{\co}{{\it (o) }}
\newcommand{\cp}{{\it (p) }}
\newcommand{\cq}{{\it (q) }}


%-------------------------------------------------------------------------------
\begin{document}
%-------------------------------------------------------------------------------

%don't want date printed
\date{}

% make title bold and 14 pt font (Latex default is non-bold, 16 pt)
\title{\Large \bf Nekton: Middlebox Analysis for Network Verification\\
CS 523 Final Report}

%for single author (just remove % characters)
\author{
{\rm Kuan-Yen Chou}
\qquad
{\rm Bin-Chou Kao}\\
University of Illinois at Urbana-Champaign
% copy the following lines to add more authors
% \and
% {\rm Name}\\
%Name Institution
} % end author

\maketitle

%-------------------------------------------------------------------------------
\begin{abstract}
%-------------------------------------------------------------------------------
%Your abstract text goes here. Just a few facts. Whet our appetites.
%Not more than 200 words, if possible, and preferably closer to 150.
In this report, we present the current progress and changes on the Nekton
project about middlebox analysis, current design of analysis procedure,
experiment, and the challenges we have encountered.
\end{abstract}


%-------------------------------------------------------------------------------
\section{Introduction}
%-------------------------------------------------------------------------------

% background
Networks play an essential role in modern computing systems, many of which are
interconnected and administered by various organizations and institutions. With
the core architecture and protocols being devised in the 1990s and early 2000s,
it has been a challenging yet important task to efficiently and effectively
manage the legacy/traditional networks without causing any serious outages
\footnote{In this paper, we use the terms "legacy networks" and "traditional
networks" interchangeably, in contrast to the newer implementations of
software-defined networks (SDN). Nonetheless, the middlebox analysis techniques
should not be limited to any certain kinds of networks.},
since a minor configuration error by a human operator can potentially affect the
behavior of multiple networks globally, which could cause serious financial
losses. As a result, it would be extremely beneficial if we are able to either
formally verify or test the correctness of a given network to some satisfactory
extent.

%problem statement
In order to verify or test a given network, one of the biggest obstacles is the
existence of middleboxes. Middleboxes are special devices that are deployed
somewhere in a network for some complicated network functions, such as packet
processing, packet filtering, traffic engineering, or packet inspection. They
can be real appliances one bought from a device vendor, or distributed as
software running on other machines. Some common middleboxes are firewalls (FWs),
network address translators (NATs), load balancers (LBs), caching proxies, and
intrusion detection/prevention systems (IDS/IPS).

For verifying networks with middleboxes, it is very hard to automatically or
efficiently create accurate models for all kinds of middleboxes. Plankton-neo
\cite{2018-PrabhuEtAl} tried to solve the problem by combining model-checking
verification with emulation. However, in Plankton-neo, for each execution path
of a network system, the corresponding representative packet of a packet
equivalence class (PEC) is injected only once, which does not provide sufficient
coverage if a middlebox behaves non-deterministically (i.e. for the same input
packet, it has more than one different possible actions, without internal state
changes). Some examples are load balancers and dynamic NATs. In addition, the
PECs are computed from the configurations of the middleboxes and other nodes,
which is not ideal in practice for two reasons. One is that it is very likely
that we don't have a formal specification of how a middlebox works in detail.
The other reason is that even if we have a detailed, formal specification of a
middlebox, it is highly probable that there are implementation bugs in the
software.
% goal
Thus, in this project, we plan to analyze some given middleboxes so as to make
it easier to verify such networks. The goal is to explore all the non-deterministic
actions given a middlebox and an input.


% vim: set ft=tex :

\section{Background}

In this section, we will introduce several middleboxes and binary analysis tools we have surveyed in this project.
In particular, three middleboxes \ca HAProxy, \cb Nginx and \cc Snort are discussed in this section.
In addition, two binary analysis tools \ca BAP and \cb Angr are introduced below.

\subsection{Middleboxes}
Middleboxes are a set of network devies that widely use in various network design.
All these divices provides different behavior for controling packets rather than packet forwarding in the middle of the network.
According to RFC 3234~\cite{rfc3234}, middleboxes can be category into several classes, including firewall, NAT, load balancer, \etc.
We tend to make our survey fit in as much classes as we can so we may get more concrete idea on how to analyze the non-deterministic behavior of them.

\subsubsection{HAProxy}
HAProxy is a connection-terminating load balancer, which means that an HAProxy instance will terminate an incoming TCP connection and re-initiate the connection on the other side(s), or drop the connection, according to the configuration. There are essentially two modes in HAProxy, TCP and HTTP. When configured in TCP mode, HAProxy would still terminate the TCP connection locally and start new connections to the backend servers, which hence normalizes the TCP packets (i.e., discarding or replacing those out-of-order packets or packets with abnormal flags). The difference between the TCP and HTTP mode is that in HTTP mode, HAProxy would inspect the layer 7 header and can be configured using those HTTP information in the HTTP request header (e.g., HTTP request URI), while in TCP mode, HAProxy will only look at the layer 3 and 4 information, which can be useful for better performance or encrypted payload like HTTPS without keys for decryption.

The non-deterministic behaviors of HAProxy given a concrete configuration can be defined as the set of possible outcomes or actions that cannot be deterministically modelled by only knowing the input request. From the documentation of the current stable release (2.0.8), the supported load balancing schemes include round robin, least connections first, first available first, source address hashing, HTTP URI hashing, HTTP URI parameters, and random.

\subsubsection{Nginx}
Nginx plays two roles in network: reverse proxy and load balancer.
These two roles are not independent.
For the role of reverse proxy, two modules are used: \ca Proxy module and \cb rewrite module.
Possible non-deterministic actions from these two modules are:
\begin{itemize}
	\item proxy\_set\_header: redefine the value of certain field in the header part of http request.
	\item proxy\_set\_body: redefine the request body passed to the proxied server.
	\item proxy\_redirect: set the text that should be changed in the “Location” and “Refresh” header fields of a proxied server response (always come with 30X response code).
	\item proxy\_pass: Set the protocol and address of a proxied server and an optional URI to which a location should be mapped.
	\item rewrite: redirect; changes the URI in the header.
\end{itemize}
Among the above, three actions, ‘proxy\_set\_header‘, ‘rewrite’ and ‘proxy\_pass’ are most possiblely to affect PEC.
For the role of load balancer, Upstream module is used.
After analyze the module, we find one ground truth for load balancer is that all load balancer will modify the packet header.
However, the forwarding behavior is not affected.
As Nginx only perform HTTP load balancing based on reverse proxy, so http headers might be modified during this process based on the algorithms user choose.
For example, hash methods will make sure packets from certain IP can only be forwarding to certain backend server.
While round-robin or fair method will distribute all the packets to different backend servers.
Thus, checking the modified HTTP packets can be a reasonable direction for analyzing non-deterministic behaviors.

\subsubsection{Snort}
Snort is an intrusion detection and prevention system (IDS/IPS), which is usually placed near a firewall as an additional protection for untrusted traffic.
There are several modes provided by Snort which simply swich the settings of Snort easily for user.
For example, Sniffer mode and Packet logger mode are two modes that don’t change PEC.
NIDS mode, on the other hand, is much interesting since it performs detection and analysis on network traffic.
Apart from acting as a stateless firewall, Snort’s "\textbf{flow}" keyword, used in conjunction with session tracking, allows Snort to track and take action on tcp packets depending on the tcp session state and its flag value.
This functionality gives a network operator very finegrained control on Snort’s behavior.

\subsection{Binary Analysis Tools}
Binary analysis tools are used for analyzing bianry codes.
Since there are various middleboxes are provided under copyright or without open srouce, it is reasonable to assume the binary code is the best source we can use for analyzing the middleboxes.
Two binary analysis tools are targeted to be the candidate for this project.

\subsubsection{BAP}
BAP is a program analysis platform written inOCaml, but also comes with Python and C binding libraries.
It basically disassembles the binary code andlifts it into their own RISC-like BAP Instruction Language (BIL) representation, on which program analysis is performed. 
We have explored a few of BAP’s core functions, such as constructing controlflow graph (CFG) and call graph (CG) given a program binary.
Control flow graph is a directed graph where eachnode is a code block of sequential instructions without branches, except for the beginning and the end of the block, and each edge is a branch from one block toanother.
Call graph,  nonetheless,  is a directed graphconsisting of code blocks of functions with edges be-ing function calls or returning to the caller function.

\subsubsection{Angr}


%-------------------------------------------------------------------------------
\section{Design and Methods}
%-------------------------------------------------------------------------------

In this project, our goal is to find function \(f\), \(\forall input stream s_i, \exists output stream s_o which s_o = f(s_i)\).
As most of the behavior in the network are predictable, we will mainly focusing on the non-deterministic action of middleboxes.
In this section, we will discribe how we design our methodology of finding output stream based on provided input stream.

\subsection{Assumptions on middleboxes}

% describe the middleboxes implementation (connection terminating and
% forwarding)
After the survey of various open source middleboxes, we observed that, in
general, the middleboxes can be categorized into two classes. One of them is the
middleboxes that would forward the transport layer header (assuming that the
end-to-end application is based on some connection-oriented protocol, like TCP),
such as NetFilter subsystem in Linux and its derivative utilities (e.g.,
iptables, nftables). These middleboxes might still forward packets to different
next hops or modify the packet headers, but they will not modify the packets so
that it breaks the transport layer connection between the client and the server.
A common example is using iptables to configure the NetFilter subsystem as a
firewall and/or NAT, in which case the layer 2 and layer 3 headers are modified,
but layer 4 header fields remain untouched. The other category consists of the
middleboxes that terminate the incoming connection at one end and start a new
connection to the backend server at another. Such middleboxes usually act as a
proxy or gateway, such as TCP proxies, HTTP proxies, SOCKS proxies (layer 5),
etc. To simplify the problem, we will only focus on the second kind of
middleboxes, which would accept/terminate the incoming connection and initiate a
new one, so that we treat each connection as request and reply application-layer
packets.

\subsection{Binary Ananlysis}
%-----------------------------------

% intro to angr, its abilities
%   CFG, block, node
%   CFG traversal
%   looking for specific instruction (like certain function call) in a block
%   looking for variables (either in registers or memory)
%   symbolic solver engine
% how we plan to use angr for binary analysis
In previous section, we discuss two open source binary analysis tools, BAP and Angr.
Since our goal is to get output stream from provided input stream, and we try to targeting non-deterministic behavior, it is reasonable to use symbolic execution by binary analysis tools to get results we need.
In addition, based on the assumption of middleboxes and our middlebox survey, load balancer seems to fit our goal.
Although there are two load balancer in our survey list, HAProxy is the better choice since it only focus on load balance function.
However, after we use both tools to analyze the HAProxy, we face two obstacles.
\ci For using BAP, it is very hard for us to do symbolic execution on it because BAP mainly use OCaml but we are lacking expierence of it.
Thus, it is painful to trace the memory location of each code and put exact values on it.
\cii While analyzing the HAProxy, it is difficult to idientify function and commands from binary code since the original structure of HAProxy is complicated.
Espetially, HAProxy provides several load balance algorithms, to idientify and analyze each of them is time-comsuming and plainful.

As the result, We decide to write a simple load balancer based on only three most frequently used load balance algorithm, \ca round-robin, \cb least-connection-first and \cc source-based hashing.
The detail and the implementation of this load balancer is introduced in following section.

\subsection{Simple Load Balancer}
%-----------------------------------

% design overview:
% classic listen, accept, fork, wait; connection endpoints proxying
As mentioned above, we have a simplified implementation of a load balancer
middlebox, which including three load balance algorithm.
The main goal for building this simple load balancer is to simplify the analysis procesure.
Thus, only the basic functions and load balance algorithms are included in the code.
We even use C rather than C++ to wirte the code since object-oriented structure will increase the difficulty of the symbolic execution.
In addition, we drop the input file and hardcode the server list and selected algorithm to avoid the infinate loop when processing file reader by using binary analysis tool.

Three algorithms we used in this code are introduced below:
\ci round-robin: This algorithm select the target server by the order of server assigned previously.
The algorithm tend to get the next candidate in order unless it is the end of the server list.
If it is the end of the server list, then the first server in server list will be provided to next request.
\cii least-connection-first: This algorithm will select the server with the least conncetion.
In order to do so, the load balancer need to store the conncetion of each server.
Since servers can only get conncetion from load balancer, by our assumption, load balancer can simply count the number of conncetion by adding one whenever it assigned a connection to a server.
\ciii source-based hashing: This algorithm will use a hash function to map the client to server. Thus, each client will always conncet to the same server base on the hash function no matter how many request it send to server.

% load-balancing schemes
% how we keep states of connections and servers

% vim: set ft=tex :

%-------------------------------------------------------------------------------
\section{Implementation}
%-------------------------------------------------------------------------------

\subsection{Simple Load Balancer}
%-----------------------------------


% vim: set ft=tex :

\section{Evaluation}

%-------------------------------------------------------------------------------
\section{Related Work}
%-------------------------------------------------------------------------------

In this section, we discuss and compare our method with three other papers,
Plankton~\cite{2020-PrabhuEtAl}, Plankton-neo~\cite{2018-PrabhuEtAl}, and
Alembic~\cite{2019-MoonEtAl}. These papers are about network verification, or
trying to solve some issues emerged from network verification. However, each of
those targets a different problem and goal.

\textbf{Plankton} is a configuration verifier that manages the large header
space by using equivalence partitioning and explores protocol executions with
explicit-state model checking. It is the first prototype of a network verifier
that achieves high scalability using model checking. However, it uses
hand-written models for network components, assumes the correctness of those
models and does not handle the case where we may not have accurate models.

\textbf{Plankton-neo} is a hybrid approach combining Plankton with emulation,
with an aim to solve the problem that we do not have accurate models for some
network components, such as software network functions. Instead of writing a new
model for each middlebox in a network, it uses emulation to guarantee the
correctness of the middlebox's behavior. However, since Plankton-neo uses
emulations for middleboxes, it cannot handle the non-determinism inside the
middleboxes themselves (i.e., it may have multiple possible outputs given the
same input and the same current state, which are not fully explored), which
means that it cannot provide guaranteed coverage.

\textbf{Alembic} is a newly proposed model inference method. It treats network
functions as blackboxes, constructs symbolic rules from configuration
documentations at the offline stage, extends the L* algorithm to infer symbolic
models in the form of finite state machines, which essentially injects packets
that would exercise their symbolic rules, and then collects the library of
symbolic models for the online stage, where they plug in concrete configurations
to generate a corresponding concrete model. The generated model has improved
coverage compared to emulations, and also has comparable accuracy, although
there are not actually any ground truth to evaluate that. However, the Alembic
approach requires a lot of human effort at the offline stage, and can only be
applied on certain type of network function implementations. We expect to have a
more universal and applicable solution with better coverage/accuracy with binary
analysis, which is not completely blackbox testing.


%-------------------------------------------------------------------------------
% vim: set ft=tex :

%-------------------------------------------------------------------------------
\section{Conclusion and Future Work}
%-------------------------------------------------------------------------------

% summary of the work and future plans
We propose a new approach of model extraction from software network functions
using binary analysis. The results show the possibility of getting an accurate
model, in spite of some limitations in terms of scalability. In the future, we
plan to explore more in this direction by learning more about binary analysis
and source-code analysis, to be able to generate models with less human
intervention.


%-------------------------------------------------------------------------------
% vim: set ft=tex :


%-------------------------------------------------------------------------------
%\section*{Acknowledgments}
%-------------------------------------------------------------------------------

%-------------------------------------------------------------------------------
%\section*{Availability}
%-------------------------------------------------------------------------------

%USENIX program committees give extra points to submissions that are
%backed by artifacts that are publicly available. If you made your code
%or data available, it's worth mentioning this fact in a dedicated
%section.

%-------------------------------------------------------------------------------
\bibliographystyle{plain}
\bibliography{\jobname,other}

%%%%%%%%%%%%%%%%%%%%%%%%%%%%%%%%%%%%%%%%%%%%%%%%%%%%%%%%%%%%%%%%%%%%%%%%%%%%%%%%
\end{document}
%%%%%%%%%%%%%%%%%%%%%%%%%%%%%%%%%%%%%%%%%%%%%%%%%%%%%%%%%%%%%%%%%%%%%%%%%%%%%%%%

%%  LocalWords:  endnotes includegraphics fread ptr nobj noindent
%%  LocalWords:  pdflatex acks
