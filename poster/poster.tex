%%%%%%%%%%%%%%%%%%%%%%%%%%%%%%%%%%%%%%%%%
% Dreuw & Deselaer's Poster
% LaTeX Template
% Version 1.0 (11/04/13)
%
% Created by:
% Philippe Dreuw and Thomas Deselaers
% http://www-i6.informatik.rwth-aachen.de/~dreuw/latexbeamerposter.php
%
% This template has been downloaded from:
% http://www.LaTeXTemplates.com
%
% License:
% CC BY-NC-SA 3.0 (http://creativecommons.org/licenses/by-nc-sa/3.0/)
%
%%%%%%%%%%%%%%%%%%%%%%%%%%%%%%%%%%%%%%%%%

%----------------------------------------------------------------------------------------
%	PACKAGES AND OTHER DOCUMENT CONFIGURATIONS
%----------------------------------------------------------------------------------------

\documentclass[final,hyperref={pdfpagelabels=false}]{beamer}

\usepackage[orientation=landscape,size=custom,width=86.4,height=55.9,scale=1.1]{beamerposter} % Use the beamerposter package for laying out the poster with a portrait orientation and an a0 paper size

\usetheme{I6pd2} % Use the I6pd2 theme supplied with this template

\usepackage[english]{babel} % English language/hyphenation

\usepackage{amsmath,amsthm,amssymb,latexsym} % For including math equations, theorems, symbols, etc

%\usepackage{times}\usefonttheme{professionalfonts}  % Uncomment to use Times as the main font
\usefonttheme[onlymath]{serif} % Uncomment to use a Serif font within math environments

\boldmath % Use bold for everything within the math environment

\usepackage{booktabs} % Top and bottom rules for tables

\graphicspath{{figures/}} % Location of the graphics files

\usecaptiontemplate{\small\structure{\insertcaptionname~\insertcaptionnumber: }\insertcaption} % A fix for figure numbering

%-------------------------------------------------------------------------------
%   CODE INCLUSION CONFIGURATION
%-------------------------------------------------------------------------------

\usepackage{listings}

%\lstset{language=C,
%        frame=LR,
%        belowcaptionskip=1\baselineskip,
%        breaklines=true,
%        xleftmargin=\parindent,
%        showstringspaces=false,
%        basicstyle=\footnotesize\ttfamily,
%        keywordstyle=\bfseries\color{Green},
%        commentstyle=\color{Gray},
%        identifierstyle=\color{Black},
%        stringstyle=\color{Orange},
%        %numbers=left, % Line numbers on left
%        %firstnumber=1, % Line numbers start with line 1
%        %numberstyle=\scriptsize\ttfamily\color{Brown},
%}

%----------------------------------------------------------------------------------------
%	TITLE SECTION
%----------------------------------------------------------------------------------------

\title{\huge Nekton: Middlebox Analysis for Network Verification} % Poster title

\author{Kuan-Yen Chou and Bin-Chou Kao} % Author(s)

\institute{CS @ UIUC} % Institution(s)

%----------------------------------------------------------------------------------------
%	FOOTER TEXT
%----------------------------------------------------------------------------------------

\newcommand{\leftfoot}{https://github.com/kyechou/nekton} % Left footer text

\newcommand{\rightfoot}{kychou2@illinois.edu} % Right footer text

%----------------------------------------------------------------------------------------

\begin{document}

\addtobeamertemplate{block end}{}{\vspace*{1ex}} % White space under blocks

\begin{frame}[t,fragile] % The whole poster is enclosed in one beamer frame

\begin{columns}[t] % The whole poster consists of two major columns, each of which can be subdivided further with another \begin{columns} block - the [t] argument aligns each column's content to the top

\begin{column}{.02\textwidth}\end{column} % Empty spacer column

\begin{column}{.465\textwidth} % The first column

%----------------------------------------------------------------------------------------
%	TERMINOLOGY
%----------------------------------------------------------------------------------------

\begin{block}{Terminology}

\begin{itemize}
    \item \textbf{Middlebox (or network function)}: an intermediary device
        performing functions other than traditional packet forwarding on
        datagram path between a source host and a destination host.
        (E.g., load balancer, cache proxy, firewall, NAT, IDS, IPS, ...)
\end{itemize}

\end{block}

%----------------------------------------------------------------------------------------
%	OBJECTIVES
%----------------------------------------------------------------------------------------

\begin{block}{Objectives}

\begin{enumerate}
    \item Given a software network function, with input stream $s_i$ and output
        stream $s_o$, we would like to find $f$, such that $s_o = f(s_i)$.
    \item With this input-output mapping $f$, essentially a model of the
        middlebox, we are able to compute outputs from certain input, or valid
        inputs from certain output.
\end{enumerate}

\end{block}

%----------------------------------------------------------------------------------------
%	MOTIVATION
%----------------------------------------------------------------------------------------

\begin{block}{Motivation}

\begin{itemize}
    \item Lack of accurate models of network functions when verifying networks
        with such devices.
        \begin{itemize}
            \item Lack of standards and implementations vary across venders.
            \item Implementation bugs/quirks not being modelled.
        \end{itemize}
    \item Binary instructions provide a low-level, faithful description of the
        input and output streams.
\end{itemize}

\end{block}

%----------------------------------------------------------------------------------------
%	METHODS
%----------------------------------------------------------------------------------------

\begin{block}{Methods}

\begin{itemize}
    \item We use angr as the main binary analysis framework and apply symbolic
        execution on variables (registers and memory areas) of the given target
        program binary.
\end{itemize}

\begin{columns}

\begin{column}{.43\textwidth}
\begin{enumerate}
    \item Build the control flow graph and call graph of the binary.
    \item Locate the instructions calling functions that read inputs or decide
        outputs.
    \item Symbolically execute the program from the initial state.
    \item Set the symbolic or concrete values for the input and output variables
        and add constraints according to the needs.
    \item Get the outputs after the function deciding the outputs, and the
        outputs would be the $f$ we want.
    \item We can use solver engine to compute valid concrete values.
\end{enumerate}
\end{column}

\begin{column}{.54\textwidth}
\begin{lstlisting}[language=python,basicstyle=\footnotesize\ttfamily]
...
accept_ins = find_ins(call_accept)
decision_ins = find_ins(call_select_server)
...
sm.explore(find=accept_ins)
s1 = sm.found[0]
...
s1.mem[cli_addr_p].short      = 2
s1.mem[cli_addr_p+2].uint16_t = 12345
s1.mem[cli_addr_p+4].uint32_t = 0x7f000001
...
sm.explore(find=decision_ins)
s2 = sm.found[0]
addr = s2.regs.rax & 0xffffffff
port = (s2.regs.rax & 0xffffffff00000000) >> 32
print(`Addr:', addr)
print(`Port:', port)
print(`Evaluated addr:', s2.solver.eval(addr))
print(`Evaluated port:', s2.solver.eval(port))
\end{lstlisting}
\end{column}

\end{columns}
\end{block}

%----------------------------------------------------------------------------------------

\end{column} % End of the first column

\begin{column}{.02\textwidth}\end{column} % Empty spacer column

\begin{column}{.47\textwidth} % The second column

%----------------------------------------------------------------------------------------
%	RESULTS
%----------------------------------------------------------------------------------------

\begin{block}{Evaluation Results}

\begin{itemize}
    \item We built a small load balancer as an example target, with two load
        balancing schemes, source-based hashing and round-robin.
\end{itemize}

\begin{columns}
\begin{column}{.48\textwidth}
\begin{lstlisting}[language=python,basicstyle=\footnotesize\ttfamily]
$ python sourcehash/solve.py --input-addr 0x7f000001 --input-port 11111
cli_addr: 0x7f000001
cli_port: 11111
Addr: <BV64 0x7f000001>
Port: <BV64 0x2328>
Evaluated cli_addr: 0x7f000001
Evaluated cli_port: 11111
Evaluated addr: 0x7f000001
Evaluated port: 9000
\end{lstlisting}
\end{column}
\begin{column}{.40\textwidth}
\end{column}
\end{columns}

\begin{columns}
\begin{column}{.48\textwidth}
\begin{lstlisting}[language=python,basicstyle=\footnotesize\ttfamily]
$ python roundrobin/solve.py --output-addr 0x7f000001 --output-port 9003
cur_iter: <BV32 cur_iter_19_32>
Addr: <BV64 (AST with symbolic variable cur_iter_19_32)>
Port: <BV64 (AST with symbolic variable cur_iter_19_32)>
Evaluated cur_iter: 3
Evaluated addr: 0x7f000001
Evaluated port: 9003
\end{lstlisting}
\end{column}
\begin{column}{.40\textwidth}
\end{column}
\end{columns}

\end{block}

%----------------------------------------------------------------------------------------
%	PERFORMANCE
%----------------------------------------------------------------------------------------

\begin{block}{Evaluation Performance}

\begin{columns}
\begin{column}{.39\textwidth}
\begin{itemize}
    \item The method works within reasonable resource consumption after pruning
        unneeded code sections.
    \item Time and memory usage increased by 13x and 58x just for parsing
        arguments.
\end{itemize}
\end{column}

\begin{column}{.58\textwidth}
\begin{table}
\begin{tabular}{l r r}
\toprule
\textbf{Experiment} & \textbf{Time} & \textbf{Memory}\\
\midrule
Source-based hashing        & 2.90 sec & 129.278 MB \\
Round-robin                 & 2.99 sec & 129.033 MB \\
Find accept w/o arg parsing & 25.06 sec & 207.260 MB \\
Find accept w/ arg parsing   & 333.40 sec & 12064.724 MB \\
\bottomrule
\end{tabular}
\caption{Performance result}
\end{table}
\end{column}
\end{columns}

%\begin{figure}
%\includegraphics[width=0.8\linewidth]{placeholder.jpg}
%\caption{Figure caption}
%\end{figure}

\end{block}

%----------------------------------------------------------------------------------------
%	LIMITATIONS AND ASSUMPTIONS
%----------------------------------------------------------------------------------------

\begin{block}{Limitations and Assumptions}

\begin{itemize}
\item Still need some human intervention and understanding of the program.
\item Current binary analysis tools are not mature enough to handle complicated
    real-world software. A lot of library functions are not properly simulated.
\end{itemize}

\end{block}

%----------------------------------------------------------------------------------------
%	CONCLUSION
%----------------------------------------------------------------------------------------

\begin{block}{Conclusion}

\begin{itemize}
    \item Binary analysis on network functions shows the possibility of getting
        an accurate model. In the future, we plan to explore further about
        source-level analysis to be able to generate models with less human
        intervention.
\end{itemize}

\end{block}

%----------------------------------------------------------------------------------------
%	REFERENCES
%----------------------------------------------------------------------------------------

%\begin{block}{References}
%
%%\nocite{*} % Insert publications even if they are not cited in the poster
%\small{\bibliographystyle{unsrt}
%\bibliography{poster}}
%
%\end{block}

%----------------------------------------------------------------------------------------
%	ACKNOWLEDGEMENTS
%----------------------------------------------------------------------------------------

%\begin{block}{Acknowledgments}
%
%\begin{itemize}
%\item Nam mollis tristique neque eu luctus.
%\end{itemize}
%
%\end{block}

%----------------------------------------------------------------------------------------
%	CONTACT INFORMATION
%----------------------------------------------------------------------------------------

%\setbeamercolor{block title}{fg=black,bg=orange!70} % Change the block title color
%
%\begin{block}{Contact Information}
%
%\begin{itemize}
%\item Web: \href{http://www.university.edu/smithlab}{http://www.university.edu/smithlab}
%\item Email: \href{mailto:john@smith.com}{john@smith.com}
%\item Phone: +1 (000) 111 1111
%\end{itemize}
%
%\end{block}

%----------------------------------------------------------------------------------------

\end{column} % End of the second column

\begin{column}{.015\textwidth}\end{column} % Empty spacer column

\end{columns} % End of all the columns in the poster

\end{frame}
% End of the enclosing frame

\end{document}
